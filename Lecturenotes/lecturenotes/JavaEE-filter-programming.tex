\chapter{Java EE过滤器编程}
\label{chp:JavaEE-filter-programming}

\section*{基本信息}
\sline
\begin{description}
\item[课程名称:] Java应用与开发
\item[授课教师:] 王晓东
\item[授课时间:] 第十二周
\item[参考教材:] 本课程参考教材及资料如下:
  \begin{itemize}
  \item 吕海东,张坤 编著,Java EE企业级应用开发实例教程,清华大学出版社,2010年8月
  \end{itemize}
\end{description}

\section*{教学目标}

\sline

\begin{enumerate}
\item 理解Java EE过滤器的概念。
\item 掌握Java EE过滤器的主要功能。
\item 掌握Java EE过滤器编程和配置方法。  
\end{enumerate}  

\section*{授课方式}

\sline
\begin{description}
\item[理论课:] 多媒体教学、程序演示
\item[实验课:] 上机编程
\end{description}

\newpage
\section*{教学内容}
\sline
%%%%%%%%%%%%%%%%%%%%%%%%%%%%%%%%%%%%%%%%%%%%%%%%%%%%%%%%%%%%%%
\section{过滤器概述}

\subsection{过滤器概述} 

Web开发中会频繁遇到诸如用户登录验证、处理中文Web遇到的汉字乱码等常见问
题,采用常规开发,例如实现用户登录验证需要多个在Servlet内部处理重复编写
处理代码,会带来大量的代码冗余。

自然而然我们会想到,能否将需要将重复出现的处理上述问题的代码从每个Web组
件中抽取出来,放在一个公共的地方,供所有需要这些公共功能代码的Web组件调
用呢?

在Servlet 2.3规范中引入了新的Web组件技术——{\hei 过滤器(Filter)},使上
述难题迎刃而解。

\subsection{过滤器的基本概念}

{\hei 过滤器,是对某种数据流动进行过滤处理的对象。在Java EE Web应用中,
  过滤器处理的这种数据流动就是HTTP请求数据流和响应数据流。}

\begin{itemize}
\item Filter是对HTTP请求和响应的头(Header)和体(Body)进行特殊操作
  的Web组件;
\item Filter本身不生成Web响应,只对Web的请求和响应做过滤处理。注意,这
  些操作都是在Servlet等Web组件和用户端浏览器毫不知情的情况下进行的。
\end{itemize}

\subsection{过滤器的基本功能} 

过滤器采用AOP{\bf\Red (Aspect Oriented Programming)}编程思想,使用拦
截技术,在HTTP请求和响应达到目标之前,对请求和响应的数据进行预处理。其
功能主要包括:

\begin{itemize}
\item 对HTTP请求作分析,对输入流进行预处理;
\item 阻止请求或响应的进行;
\item 根据需求改动请求头的信息和数据体;
\item 根据需求改动响应的头(Header)和体(Body)数据;
\item 与其他Web资源进行协作。
\end{itemize}

过滤器的主要应用领域包括:

\begin{itemize}
\item 登录检验
\item 权限审核
\item 数据验证
\item 日志登记
\item 数据压缩/解压缩
\item 数据加密/解密
\end{itemize}

所以,当我们在编程过程中遇到需要解决上述应用问题的时候,我们应该考虑使
用过滤器技术。

\section{Java EE过滤器API}

\subsection{Javax.servlet.Filter接口} 

所有过滤器必须实现javax.servlet.Filter接口,该接口包含以下方法:

\subsubsection{public void init(FilterConfig filterConfig) throws
  ServletException}

过滤器初始化方法,在Web容器创建过滤器对象后被调用,用于完成过滤器初始化操作,
如取得过滤器配置的参数,连接外部资源。

\subsubsection{public void doFilter(ServletRequest request,
  ServletResponse response,FilterChain chain) throws IOException,
  ServletException}

过滤器的核心方法,在满足过滤器过滤目标URL的请求和响应时调用,开发人员在
此方法中编写过滤功能代码。

\subsubsection{public void destroy()}

在过滤器销毁之前此方法被调用,此方法主要编写清理和关闭打开的资源操作,
如关闭数据库连接、将过滤信息保存到外部资源操作。

\subsection{Filter的doFilter()方法} 

过滤器的doFilter()方法参数包括:

\begin{description}
\item[参数1] 请求对象 javax.servlet.ServletRequest
\item[参数2] 响应对象 javax.servlet.ServletResponse
\item[参数3] 过滤链对象 javax.servlet.FilterChain
\end{description}

过滤器的doFilter()方法功能描述如下:

\begin{itemize}
\item 此方法在每次过滤被激活时被调用。
\item 此方法代码完成过滤器实际的逻辑操作功能。
\item 如果是HTTP请求,需要强转
  为HttpServletRequest和HttpServletResponse。
\item 经过过滤器的请求和响应对象会被传递到被过滤的JSP或Servlet。
\item 可以通过对request对象操作,在Servlet之前修改请求对象的信息。
\item 可以通过response对象操作,在Servlet响应之前修改响应信息。
\end{itemize}

\subsection{javax.servlet.FilterChain接口} 

在Java EE规范中对每个URL的请求和响应都可以定义多个过滤器,这些过滤器构
成过滤器链。FilterChain接口的对象用来表示过滤器链,
是Filter的doFilter()方法的第三个参数。

\begin{itemize}
\item 过滤器使用FilterChain接口的doFilter()方法来调用过滤器链中的下一个
  过滤器,如果没有下级过滤器,则将用doFilter方法调用末端
  的JSP和Servlet。
\item 如果在过滤器的过滤方法中不调用FilterChain的传递方法doFilter(),则
  将{\bf\Red 截断}对下级过滤器或JSP/Servlet的请求和响应,使得Web容器没
  有机会运行JSP或Servlet,达到阻断请求和响应的目的。
\end{itemize}

FilterChain的doFilter()方法定义如下:

\ttc{public void doFilter(ServletRequest request, ServletResponse
  response) throws IOException, ServletException}

\begin{itemize}
\item 此方法完成调用下级过滤器或最终请求资源,如JSP和Servlet;
\item 此方法传递请求对象和响应对象这两个参数,并将请求对象和响应对象传
  递到下级过滤器或Web组件。
\end{itemize}

\subsection{javax.servlet.FilterConfig接口} 

FilterConfig接口定义了取得过滤器配置的初始参数方法,通过实现了该接口的
对象,可以取得在配置过滤器时初始参数和ServletContext上下文对象,进而取
得Web应用信息。

\subsubsection{public String getInitParameter(String name)}

该方法取得过滤器配置的初始参数。

\subsubsection{public Enumeration getInitParameterNames()}

该方法取得过滤器配置的所有初始参数,以枚举器类型返回。

\subsubsection{public String getFilterName()}

该方法取得配置的过滤器名称。

\subsubsection{public ServletContext getServletContext()}

该方法取得过滤器运行的Web应用环境对象,通过ServletContext对象,过滤器可以取得
所有Web应用环境数据供过滤器使用。

\section{Java EE过滤器编程和配置}

完成Java EE过滤器的编程和配置包括两个步骤:

\begin{enumerate}
\item 自定义过滤器,实现javax.servlet.Filter接口,并完成相关方法的重写。
\item 在Web配置文件/WEB-INF/web.xml中完成对过滤器的配置。
\end{enumerate}

以下我们给出一个过滤器编程和配置的示例。我们定义CharEncodingFilter过滤
器,该过滤器的主要功能是基于用户配置的内容类型和编码方式,设置请求对象
的编码。

首先,我们编写过滤器类CharEncodingFilter.java,实现Filter接口的所有方法。

\samplecode{CharEncodingFilter.java}

\begin{javaCode}
  public class CharEncodingFilter implements Filter {
    private FilterConfig config = null;
    private String contentType = null;
    private String code = null;

    public void init(FilterConfig config) throws ServletException {
      this.config = config;
      contentType = config.getInitParameter("contentType");
      code = config.getInitParameter("encoding");
    }

    public void doFilter(ServletRequest req, ServletResponse res, FilterChain chain) 
    throws IOException, ServletException {
      //转换为 HTTP 请求对象
      HttpServletRequest request = (HttpServletRequest)req;
      if (request.getContentType().equals(contentType)) {
        request.setCharacterEncoding(code); // 设置字符编码集
      }
      //继续下个过滤器
      chain.doFilter(req, res);
    }

    public void destroy() {
      //放置过滤器销毁的处理代码
    }
  }
\end{javaCode}

下面,我们配置过滤器,在Web应用的配置文件/WEB-INF/web.xml中配置声明和过滤URL地址映射。

\ttc{过滤器声明}

\begin{xmlCode}
  <filter>
    <display-name>字符集编码过滤器</display-name>
    <description>此过滤器完成对请求数据字符编码的修改</description>
    <filter-name>EncodingFilter</filter-name>
    <filter-class>javaee.ch08.CharEncodingFilter</filter-class>
    <init-param>
      <description>Content Type</description>
      <param-name>contentType</param-name>
      <param-value>text/html</param-value>
    </init-param>
    <init-param>
      <description>Encoding</description>
      <param-name>encoding</param-name>
      <param-value>utf-8</param-value>
    </init-param>
  </filter>
\end{xmlCode}

\ttc{过滤器URL映射}

\begin{itemize}
\item 过滤器需要对所过滤的URL进行映射。当浏览器访问的Web文档URL地址符合
  过滤器的映射地址时,此过滤器自动开始工作。
\item 如果有多个过滤器对某个URL地址都符合时,这些过滤器构成过滤器链,先
  声明的过滤器先运行,运行的顺序与声明的次序一致。
\end{itemize}

过滤器映射语法:

\begin{xmlCode}
  <filter-mapping>
    <filter-name>EncodingFilter</filter-name>
    <servlet-name>SaveCookie</servlet-name>
    <servlet-name>GetCookie</servlet-name>
    <url-pattern>/employee/add.do</url-pattern>
    <url-pattern>/admin/*</url-pattern>
    <dispatcher>FORWARD</dispatcher>
    <dispatcher>INCLUDE</dispatcher>
    <dispatcher>REQUEST</dispatcher>
    <dispatcher>ERROR</dispatcher>
  </filter-mapping>
\end{xmlCode}


对过滤器映射标记的说明:

\begin{description}
\item[<filter-mapping>] 与<filter>标记平级,且在<filter>之后,即先声明
  后映射的原则。
\item[<filter-name>] 应该与过滤器声明中的<filter-name>一致。
\item[<servlet-name>] 指示过滤器对指定的Servlet进行过滤,每个过滤器映射
  可以定义多个。
\item[<url-pattern>] 过滤器映射地址声明,每个过滤器映射可以定义多个。
\item[<dispatcher>] 从Servlet API 2.4开始,过滤器映射增加了根据请求类型
  有选择的对映射地址进行过滤,提供标记<dispatcher>实现请求类型的选择。
  \begin{itemize}\kai
  \item REQUEST 当请求直接来自客户时,过滤器才工作。
  \item FORWARD 当请求是来自Web组件转发到另一个组件时,过滤器工作。
  \item INCLUDE 当请求来自include操作时,过滤器生效。
  \item ERROR 当转发到错误页面时,过滤器起作用。
  \end{itemize}
\end{description}

\subsection{过滤器生命周期} 

\begin{enumerate}
\item {\hei\Blue 创建阶段} 根据<filter-class>标记定义的过滤器类,将类定
  义加载到服务器内存,并调用此类的默认构造方法,创建过滤器对象。
\item {\hei\Blue 初始化阶段} 创建FilterConfig对象,调用过滤器的init()方
  法,传入FilterConfig象,完成初始化工作。
\item {\hei\Blue 过滤服务阶段} 每次请求符合过滤器配置的URL时,过滤方法
  都将执行一次。
\item {\hei\Blue 销毁阶段} 当Web应用卸载或Web容器停止之前,destroy()方法
  被Web容器调用,完成卸载操作,Web容器销毁过滤器对象。
\end{enumerate}

\section{过滤器的主要任务}

\subsection{处理HTTP请求} 

\tta{修改请求头}

调用ServletRequest或HttpServletRequest的各种set请求头方法对请求头进行修
改。

\begin{javaCode}
  request.setCharacterEncoding("code");
\end{javaCode}

{\kai 对文字乱码问题,可以在过滤器中进行集中处理,避免在每
  个Servlet或JSP中进行请求字符编码的转换。}

\tta{修改请求对象的属性}

调用请求对象的setAttribute方法对请求对象的属性进行增加、修改和删除。

\begin{javaCode}
  request.setAttribute("infoType", "image/jpeg"); //设定请求对象的一个属性
  request.removeAttribute("userId");
\end{javaCode}

\subsection{处理HTTP响应} 

过滤器可以在HTTP响应到达客户端浏览器之前,对响应头和响应体进行转换、修
改等操作。过滤器实现对响应处理的代码要在FilterChain的doFilter()方法之后
完成,而对请求处理的代码要在doFilter()方法之前进行,流程如下:

\begin{javaCode}
  public void doFilter(ServletRequest req, ServletResponse res, FilterChain chain) 
  throws IOException, ServletException {
    ...  // 处理请求的代码放在 doFilter 之后
    chain.doFilter(req, res); // 传递到链中的下个过滤器
    ...  // 处理响应的代码放在过滤器链传递之后
  }  
\end{javaCode}

\tta{修改响应头}

\begin{javaCode}
  response.setContentType("application/pdf");
  response.setContentType("GBK");
\end{javaCode}

\tta{修改响应体内容}

过滤器使用HTTP响应对象的包装
类javax.servlet.http.HttpServletResponseWrapper可以将响应体内容进行重新
包装和处理,使浏览器接收到的是经过过滤器处理修改过的响应数据。


\subsection{阻断HTTP请求} 

阻断HTTP请求常用于用户登录验证等操作。在某种条件下,实现对请求的阻断,
不让请求传递到过滤器链中的下一个对象,只要在过滤器的过滤方法中,不执
行FilterChain的传递方法doFilter()即可。

\begin{javaCode}
  if ( 某条件成立 ) {
    chain.doFilter(request, response); // 继续传递请求
  } else {
    response.sendRedirect("url"); // 阻断请求,直接重定向到指定的 URL
  }
\end{javaCode}

\section{课后习题}

\tta{简答题}

\begin{enumerate}
\item 什么是过滤器?过滤器通常用于那些场景?
\item Java EE过滤器的工作原理是怎样的?
\end{enumerate}

\tta{小编程}

\begin{enumerate}
\item 编写过滤器,结合Session等技术,实现用户登录验证。
\end{enumerate}