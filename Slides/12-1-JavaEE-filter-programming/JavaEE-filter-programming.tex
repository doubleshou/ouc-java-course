%%%%%%%%%%%%%%%%%%%%%%%%%%%%%%%%%%%%%%%%%%%%%%%%%%%%%%%%%%%%%%%%%%%%%%%%%%%%%%%
\documentclass[hyperref={pdfpagelabels=false},compress,table]{beamer} % 在Mac下无法编译
% \documentclass[compress,table]{beamer} % 在Mac下使用
% package for font
\usepackage{fontspec}
\defaultfontfeatures{Mapping=tex-text}  %%如果没有它,会有一些 tex 特殊字符无法正常使用,比如连字符。
\usepackage{xunicode,xltxtra}
\usepackage[BoldFont,SlantFont,CJKnumber,CJKchecksingle]{xeCJK}  % \CJKnumber{12345}: 一万二千三百四十五
\usepackage{CJKfntef}  %%实现对汉字加点、下划线等。
\usepackage{pifont}  % \ding{}
% package for math
\usepackage{amsfonts}

% package for graphics
\usepackage[americaninductors,europeanresistors]{circuitikz}
\usepackage{tikz}
\usetikzlibrary{plotmarks}  % placements=positioning
\usepackage{graphicx}  % \includegraphics[]{}
\usepackage{subfigure}  %%图形或表格并排排列
% package for table
\usepackage{colortbl,dcolumn}  %% 彩色表格
\usepackage{multirow}
\usepackage{multicol}
\usepackage{booktabs}
% package for code
\usepackage{fancyvrb}
\usepackage{listings}

% \usepackage{animate}
% \usepackage{movie15}

%%%%%
% setting for beamer
\usetheme{default} % Madrid(常用), Copenhagen, AnnArbor, boxes(白色), Frankfurt,Berkeley
\useoutertheme[subsection=true]{miniframes} % 使用Berkeley时注释本行
\usecolortheme{sidebartab}
\usefonttheme{serif}  %%英文使用衬线字体
% \setbeamertemplate{background canvas}[vertical
% shading][bottom=white,top=structure.fg!7] %%背景色,上25%的蓝,过渡到下白。
\setbeamertemplate{theorems}[numbered]
\setbeamertemplate{navigation symbols}{}  %% 去掉页面下方默认的导航条
\setbeamercovered{transparent}  %设置 beamer 覆盖效果

% 设置标题title背景色
% \setbeamercolor{title}{fg=black, bg=lightgray!60!white}
\setbeamercolor{title}{fg=white, bg=black!70!white}

% 设置每页小LOGO
\pgfdeclareimage[width=1cm]{ouc}{figures/static/ouc.pdf}
\logo{\pgfuseimage{ouc}{\vspace{-20pt}}}

% Setting for font
%%\setCJKmainfont{Adobe Kaiti Std}
%%\setCJKmainfont{SimSun} 
%% \setCJKmainfont{FangSong_GB2312} 
%% \setmainfont{Apple Garamond}  %%苹果字体没有SmallCaps
\setCJKmainfont{SimSun}
%%\setCJKmainfont{Adobe Song Std}
%FUNNY%\setCJKmainfont{DFPShaoNvW5-GB}  %%华康少女文字W5(P)
%FUNNY%\setCJKmainfont{FZJingLeiS-R-GB}  %%方正静蕾体
%FUNNY%\setmainfont{Purisa}
%\setsansfont[Mapping=tex-text]{Adobe Song Std}
     %如果装了Adobe Acrobat,可在font.conf中配置Adobe字体的路径以使用其中文字体。
     %也可直接使用系统中的中文字体如SimSun、SimHei、微软雅黑等。
     %原来beamer用的字体是sans family;注意Mapping的大小写,不能写错。
     %设置字体时也可以直接用字体名,以下三种方式等同:
     %\setromanfont[BoldFont={黑体}]{宋体}
     %\setromanfont[BoldFont={SimHei}]{SimSun}
     %\setromanfont[BoldFont={"[simhei.ttf]"}]{"[simsun.ttc]"}

     % setting for graphics
\graphicspath{{figures/}}  %%图片路径
\renewcommand\figurename{图}

% setting for pdf
\hypersetup{% pdfpagemode=FullScreen,%
            pdfauthor={Xiaodong Wang},%
            pdftitle={Title},%
            CJKbookmarks=true,%
            bookmarksnumbered=true,%
            bookmarksopen=false,%
            plainpages=false,%
            colorlinks=true,%
            citecolor=green,%
            filecolor=magenta,%
            linkcolor=blue,%red(default)
            urlcolor=cyan}

% setting for fontspec
\XeTeXlinebreaklocale "zh"  %%表示用中文的断行
\XeTeXlinebreakskip = 0pt plus 1pt minus 0.1pt  %%多一点调整的空间
%%%%%

% Font setting by xeCJK
\setCJKfamilyfont{NSimSun}{NSimSun}
\newcommand{\song}{\CJKfamily{NSimSun}}
%%%\setCJKfamilyfont{AdobeSongStd}{Adobe Song Std}
%%%\newcommand{\AdobeSong}{\CJKfamily{AdobeSongStd}}
\setCJKfamilyfont{FangSong}{FangSong_GB2312}
\newcommand{\fang}{\CJKfamily{FangSong}}
%%%\setCJKfamilyfont{AdobeFangsongStd}{Adobe Fangsong Std}
%%%\newcommand{\AdobeFang}{\CJKfamily{AdobeFangsongStd}}
%%%\setCJKfamilyfont{SimHei}{SimHei}
\setCJKfamilyfont{SimHei}{Adobe Heiti Std}
\newcommand{\hei}{\CJKfamily{SimHei}}
%%%\setCJKfamilyfont{AdobeHeitiStd}{Adobe Heiti Std}
%%%\newcommand{\AdobeHei}{\CJKfamily{AdobeHeitiStd}}
%%%\setCJKfamilyfont{KaiTi}{KaiTi}
\setCJKfamilyfont{KaiTi}{Adobe Kaiti Std}
\newcommand{\kai}{\CJKfamily{KaiTi}}
%%%\setCJKfamilyfont{AdobeKaitiStd}{Adobe Kaiti Std}
\newcommand{\AdobeKai}{\CJKfamily{AdobeKaitiStd}}
\setCJKfamilyfont{LiSu}{LiSu}
\newcommand{\li}{\CJKfamily{LiSu}}
\setCJKfamilyfont{YouYuan}{YouYuan}
\newcommand{\you}{\CJKfamily{YouYuan}}
\setCJKfamilyfont{FZJingLei}{FZJingLeiS-R-GB}
\newcommand{\jinglei}{\CJKfamily{FZJingLei}}
\setCJKfamilyfont{MSYH}{Microsoft YaHei}
\newcommand{\msyh}{\CJKfamily{MSYH}}

% 自定义颜色
\def\Red{\color{red}}
\def\Green{\color{green}}
\def\Blue{\color{blue}}
\def\Mage{\color{magenta}}
\def\Cyan{\color{cyan}}
\def\Brown{\color{brown}}
\def\White{\color{white}}
\def\Black{\color{black}}

\lstnewenvironment{xmlCode}[1][]{% for Java
  \lstset{
    basicstyle=\tiny\ttfamily,%
    columns=flexible,%
    framexleftmargin=.7mm, %
    % frame=shadowbox,%
    % rulesepcolor=\color{cyan},%
     frame=single,%
    backgroundcolor=\color{white},%
    xleftmargin=4\fboxsep,%
    xrightmargin=4\fboxsep,%
    numbers=left,numberstyle=\tiny,%
    numberblanklines=false,numbersep=7pt,%
    language=xml, %
    }\lstset{#1}}{}

\lstnewenvironment{javaCode}[1][]{% for Java
  \lstset{
    basicstyle=\tiny\ttfamily,%
    columns=flexible,%
    framexleftmargin=.7mm, %
    frame=shadowbox,%
    rulesepcolor=\color{cyan},%
    % frame=single,%
    backgroundcolor=\color{white},%
    xleftmargin=4\fboxsep,%
    xrightmargin=4\fboxsep,%
    numbers=left,numberstyle=\tiny,%
    numberblanklines=false,numbersep=7pt,%
    language=Java, %
  }\lstset{#1}}{}


\lstnewenvironment{jspCode}[1][]{% for JSP
  \lstset{
    basicstyle=\tiny\ttfamily,%
    columns=flexible,%
    framexleftmargin=.7mm, %
    % frame=shadowbox,%
    % rulesepcolor=\color{cyan},%
    frame=single,%
    backgroundcolor=\color{white},%
    xleftmargin=4\fboxsep,%
    xrightmargin=4\fboxsep,%
    numbers=left,numberstyle=\tiny,%
    numberblanklines=false,numbersep=7pt,%
    language=xml, %
  }\lstset{#1}}{}
  
\lstnewenvironment{shCode}[1][]{% for Java
  \lstset{
    basicstyle=\scriptsize\ttfamily,%
    columns=flexible,%
    framexleftmargin=.7mm, %
    frame=shadowbox,%
    rulesepcolor=\color{brown},%
    % frame=single,%
    backgroundcolor=\color{white},%
    xleftmargin=4\fboxsep,%
    xrightmargin=4\fboxsep,%
    numbers=left,numberstyle=\tiny,%
    numberblanklines=false,numbersep=7pt,%
    language=sh, %
    }\lstset{#1}}{}

\newcommand\ask[1]{\vskip 4bp \tikz \node[rectangle,rounded corners,minimum size=6mm,
  fill=white,]{\Cyan \includegraphics[height=1.5cm]{question} \Large \msyh #1};}

\newcommand\wxd[1]{\vskip 4bp \tikz \node[rectangle,minimum size=6mm,
  fill=blue!60!white,]{\White \ding{118} \msyh #1};}

\newcommand\xyy[1]{\vskip 2bp \tikz \node[rectangle,minimum size=3mm,
  fill=black!80!white,]{\White \msyh\scriptsize #1};}

\newcommand\homework[1]{\vskip 2bp \tikz \node[rectangle,minimum size=3mm,
  fill=red!80!white,]{\White \ding{45} \msyh\scriptsize 课后小作业 } ; {\kai\small #1}} 

\newcommand\cxf[1]{\vskip 4bp \tikz \node[rectangle,rounded corners,minimum size=6mm,
  fill=purple!60!white,]{\White \ding{42} \msyh #1};}

\newcommand\tta[1]{\vskip 4bp \tikz \node[rectangle,minimum size=6mm,
  fill=blue!60!white,]{\White \ding{118} \msyh #1};}

\newcommand\ttb[1]{\vskip 4bp \tikz \node[rectangle,rounded corners,minimum size=6mm,
  fill=purple!80!white,]{\White\msyh #1};}

\newcommand\ttc[1]{\vskip 2bp \tikz \node[rectangle,minimum size=3mm,
  fill=black!80!white,]{\White \msyh\scriptsize #1};}

\newcommand\notice[1]{\vskip 4bp \tikz \node[rectangle,rounded corners,minimum size=6mm,
  fill=red!80!white,]{\White \scriptsize \ding{42} \msyh #1};}

\newcommand\samp[1]{\vskip 2bp \tikz \node[rectangle,minimum size=3mm,
  fill=white!100!white,]{\Mage\msyh \small CODE \ding{231} \Black #1};\vskip -8bp}

\newcommand\codeset[1]{\vskip 2bp \tikz \node[rectangle,minimum size=3mm,
  fill=white!100!white,]{\Mage\msyh \small 课程配套代码 \ding{231} \Black #1};\vskip -8bp}

\newcommand\pptlink[2]{\vskip 4bp \tikz \node[rectangle,rounded corners,minimum size=6mm,
  fill=blue!70!white,]{\href{run:#1}{\White \scriptsize \msyh 动画演示 #2}};}



\setbeamerfont{frametitle}{series=\msyh} % 修改Beamer标题字体

\makeatletter
\newcommand{\Extend}[5]{\ext@arrow 0099{\arrowfill@#1#2#3}{#4}{#5}}
\makeatother


%%%%%%%%%%%%%%%%%%%%%%%%%%%%%%%%%%%%%%%%%%%%%%%%%%%%%%%%%%%%%%%%%%%%%%%%%%%%%%%
% \titlepage
\title[KevinW@OUC]{\hei {\huge Java应用与开发}\\  
Java EE过滤器编程}
\author[王晓东]{王晓东\\
  \href{mailto:wangxiaodong@ouc.edu.cn}{\footnotesize wangxiaodong@ouc.edu.cn}}
\institute[中国海洋大学]{\small 中国海洋大学}
\date{\today}
\titlegraphic{\vspace{-6em}\includegraphics[height=6cm]{static/ouc.pdf}\vspace{-6em}}
%%%%%
\begin{document}
%% Delete this, if you do not want the table of contents to pop up at
%% the beginning of each subsection:
\AtBeginSection[]{                              % 在每个Section前都会加入的Frame
  \frame<handout:0>{
    \frametitle{\textbf{\hei 接下来…}}
    \tableofcontents[currentsection]
  }
}  %

\AtBeginSubsection[]                            % 在每个子段落之前
{
  \frame<handout:0>                             % handout:0 表示只在手稿中出现
  {
    \frametitle{\textit{\hei 接下来…}}\small
    \tableofcontents[current,currentsubsection] % 显示在目录中加亮的当前章节
  }
}
 \frame{\titlepage}

%%%%%%%%%%%%%%%%%%%%%%%%%%%%%%%%%%%%%%%%%%%%%%%%
\begin{frame}
\frametitle{学习目标}
\begin{enumerate}
\item 理解Java EE过滤器的概念。
\item 掌握Java EE过滤器的主要功能。
\item 掌握Java EE过滤器编程和配置方法。
\end{enumerate}  
\end{frame}

\section*{大纲}
\frame{\frametitle{大纲} \tableofcontents}

\section{过滤器概述}

\begin{frame}
\frametitle{过滤器概述} 

\wxd{Web开发所遇到的常见问题}

\begin{itemize}
\item 用户登录验证
\item 开发中文Web遇到的汉字乱码
\end{itemize}

常规开发会带来大量的代码冗余,需要将处理上述问题的代码从每个Web组件中抽取出来,放在一个公共的地方,供所有需要这些公共功能代码的Web组件调用。

在Servlet 2.3规范中引入了新的Web组件技术——{\hei 过滤器(Filter)},使上述难题迎刃而解。
\end{frame}

\begin{frame}[fragile] % [fragile]参数使得能够插入代码
\frametitle{过滤器的基本概念}

{\hei 过滤器,对某种数据流动进行过滤处理的对象。在Java EE Web应用中,这种数据流动就是HTTP请求数据流和响应数据流。}

\begin{itemize}
\item Filter是对HTTP请求和响应的头(Header)和体(Body)进行特殊操作的Web组件。
\item Filter本身不生成Web响应,只对Web的请求和响应做过滤处理。这些操作都是在Web组件和浏览器毫不知情的情况下进行的。
\end{itemize}
\end{frame}

\begin{frame}[fragile] % [fragile]参数使得能够插入代码
\frametitle{过滤器的基本功能} 
过滤器采用AOP(Aspect Oriented Programming)编程思想,使用拦截技术,在HTTP请求和响应达到目标之前,对请求和响应的数据进行预处理。主要包括:
\begin{itemize}
\item 对HTTP请求作分析,对输入流进行预处理。
\item 阻止请求或响应的进行。
\item 根据需求改动请求头的信息和数据体。
\item 根据需求改动响应的头(Header)和体(Body)数据。
\item 与其他Web资源进行协作。
\end{itemize}
\end{frame}

\begin{frame}[fragile] % [fragile]参数使得能够插入代码
\frametitle{过滤器的主要应用领域} 
\begin{itemize}
\item 登录检验
\item 权限审核
\item 数据验证
\item 日志登记
\item 数据压缩/解压缩
\item 数据加密/解密
\end{itemize}
\end{frame}

\section{Java EE过滤器API}

\begin{frame}[fragile] % [fragile]参数使得能够插入代码
\frametitle{javax.servlet.Filter接口} 

所有过滤器必须实现javax.servlet.Filter接口。

\xyy{public void init(FilterConfig filterConfig) throws ServletException}

初始化方法,在Web容器创建过滤器对象后被调用,用于完成过滤器初始化操作,如取得过滤器配置的参数,连接外部资源。

\xyy{public void doFilter(ServletRequest request, ServletResponse response,FilterChain chain) throws IOException, ServletException}

过滤器的核心方法,在满足过滤器过滤目标URL的请求和响应时调用,开发人员在此方法中编写过滤功能代码。

\xyy{public void destroy()}

在过滤器销毁之前此方法被调用,此方法主要编写清理和关闭打开的资源操作,如关闭数据库连接、将过滤信息保存到外部资源操作。
\end{frame}

\begin{frame}[fragile] % [fragile]参数使得能够插入代码
\frametitle{Filter的doFilter()方法} 

\begin{description}
\item[参数1] 请求对象 javax.servlet.ServletRequest
\item[参数2] 响应对象 javax.servlet.ServletResponse
\item[参数3] 过滤链对象 javax.servlet.Filter
\end{description}

\begin{itemize}
\item 此方法在每次过滤被激活时被调用。
\item 此方法代码完成过滤器的操作功能。
\item 如果是HTTP请求,需要强转为HttpServletRequest和HttpServletResponse。
\item 过滤器的请求和响应对象会被传递到被过滤的JSP或Servlet。
\item 可以通过对request对象操作,在Servlet之前修改请求对象的信息。
\item 通过response对象操作,在Servlet响应之前修改响应信息。
\end{itemize}
\end{frame}



\begin{frame}[fragile] % [fragile]参数使得能够插入代码
\frametitle{javax.servlet.FilterChain接口} 

此接口的对象表达过滤器链,在Java EE规范中对每个URL的请求和响应都可以定义多个过滤器,这些过滤器构成过滤器链。
\begin{itemize}\kai
\item 过滤器使用FilterChain接口的doFilter()方法来调用过滤器链中的下一个过滤器,如果没有下级过滤器,则将用doFilter方法调用末端的JSP和Servlet。
\item 如果在过滤器的过滤方法中不调用FilterChain的传递方法doFilter(),则将截断对下级过滤器或JSP/Servlet的请求和响应,使得Web容器没有机会运行JSP或Servlet,达到阻断请求和响应的目的。
\end{itemize}
\end{frame}

\begin{frame}[fragile] % [fragile]参数使得能够插入代码
\frametitle{FilterChain的doFilter()方法} 

\xyy{public void doFilter(ServletRequest request, ServletResponse response) throws IOException, ServletException}

\begin{itemize}
\item 此方法完成调用下级过滤器或最终请求资源,如JSP和Servlet。
\item 该方法传递请求对象和响应对象两个参数,并将请求对象和响应对象传递到下级过滤器或Web组件,这个过滤器链就可以共用一个请求对象和响应对象。
\end{itemize}
\end{frame}

\begin{frame}[fragile] % [fragile]参数使得能够插入代码
\frametitle{javax.servlet.FilterConfig接口} 

FilterConfig接口定义了取得过滤器配置的初始参数方法,通过实现了该接口的对象,可以取得在配置过滤器时初始参数和ServletContext上下文对象,进而取得Web应用信息。

\xyy{public String getInitParameter(String name)} \\取得过滤器配置的初始参数。

\xyy{public Enumeration getInitParameterNames()} \\取得过滤器配置的所有初始参数,以枚举器类型返回。

\xyy{public String getFilterName()} \\取得配置的过滤器名称。

\xyy{public ServletContext getServletContext()} \\取得过滤器运行的Web应用环境对象,通过ServletContext对象,过滤器可以取得所有Web应用环境数据供过滤器使用。
\end{frame}

\section{Java EE过滤器编程和配置}

\begin{frame}[fragile] % [fragile]参数使得能够插入代码
\frametitle{Java EE过滤器编程和配置}
\begin{enumerate}
\item 实现javax.servlet.Filter接口,编写过滤器类。
\item 在Web配置文件/WEB-INF/web.xml中完成对过滤器的配置。
\end{enumerate}
\end{frame}

\begin{frame}[fragile] % [fragile]参数使得能够插入代码
\frametitle{过滤器编程示例} 

编写过滤器类CharEncodingFilter.java,实现Filter接口的所有方法。

\begin{javaCode}
public class CharEncodingFilter implements Filter {
  private FilterConfig config = null;
  private String contentType = null;
  private String code = null;

  public void init(FilterConfig config) throws ServletException {
    this.config = config;
    contentType = config.getInitParameter("contentType");
    code = config.getInitParameter("encoding");
  }

  public void doFilter(ServletRequest req, ServletResponse res, FilterChain chain) 
  throws IOException, ServletException {
    //转换为 HTTP 请求对象
    HttpServletRequest request = (HttpServletRequest)req;
    if (request.getContentType().equals(contentType)) {
      request.setCharacterEncoding(code); // 设置字符编码集
    }
    //继续下个过滤器
    chain.doFilter(req, res);
  }

  public void destroy() {
    //放置过滤器销毁的处理代码
  }
}
\end{javaCode}
\end{frame}

\begin{frame}[fragile] % [fragile]参数使得能够插入代码
\frametitle{过滤器编程示例} 

配置过滤器,在Web应用的配置文件/WEB-INF/web.xml中配置声明和过滤URL地址映射。

\wxd{过滤器声明}
\begin{xmlCode}
<filter>
  <description>此过滤器完成对请求数据编写及进行修改</description>
  <display-name>字符集编码过滤器</display-name>
  <filter-name>EncodingFilter</filter-name><!-- * -->
  <filter-class>javaee.ch08.CharEncodingFilter</filter-class><!-- * -->
  <init-param>
    <description>Content Type</description>
    <param-name>contentType</param-name>
    <param-value>text/html</param-value>
  </init-param>
  <init-param>
    <description>Encoding</description>
    <param-name>encoding</param-name>
    <param-value>utf-8</param-value>
  </init-param>
</filter>
\end{xmlCode}
\end{frame}

\begin{frame}[fragile] % [fragile]参数使得能够插入代码
\frametitle{过滤器编程示例} 

配置过滤器,在Web应用的配置文件/WEB-INF/web.xml中配置声明和过滤URL地址映射。

\wxd{过滤器URL映射}

\begin{itemize}
\item 过滤器需要对所过滤的URL进行映射。当浏览器访问的Web文档URL地址符合过滤器的映射地址时,此过滤器自动开始工作。
\item 如果有多个过滤器对某个URL地址都符合时,这些过滤器构成过滤器链,先声明的过滤器先运行,运行的顺序与声明的次序一致。
\end{itemize}
\end{frame}

\begin{frame}[fragile] % [fragile]参数使得能够插入代码
\frametitle{过滤器编程示例} 

过滤器映射语法:

\begin{xmlCode}
<filter-mapping>
  <filter-name>EncodingFilter</filter-name>
  <servlet-name>SaveCookie</servlet-name>
  <servlet-name>GetCookie</servlet-name>
  <url-pattern>/employee/add.do</url-pattern>
  <url-pattern>/admin/*</url-pattern>
  <dispatcher>FORWARD</dispatcher>
  <dispatcher>INCLUDE</dispatcher>
  <dispatcher>REQUEST</dispatcher>
  <dispatcher>ERROR</dispatcher>
</filter-mapping>
\end{xmlCode}
\end{frame}

\begin{frame}[fragile] % [fragile]参数使得能够插入代码
\frametitle{过滤器编程示例} 

对过滤器映射标记的说明:

\begin{description}
\item[<filter-mapping>] 与<filter>标记平级,且在<filter>之后,即先声明后映射的原则。
\item[<filter-name>] 应该与过滤器声明中的<filter-name>一致。
\item[<url-pattern>] 过滤器映射地址声明,每个过滤器映射可以定义多个。
\item[<servlet-name>] 指示过滤器对指定的Servlet进行过滤,每个过滤器映射可以定义多个。
\end{description}
\end{frame}

\begin{frame}[fragile] % [fragile]参数使得能够插入代码
\frametitle{过滤器编程示例} 

\begin{description}
\item[<dispatcher>] 
从Servlet API 2.4开始,过滤器映射增加了根据请求类型有选择的对映射地址进行过滤,提供标记<dispatcher>实现请求类型的选择。
\begin{itemize}\kai
\item REQUEST 当请求直接来自客户时,过滤器才工作。
\item FORWARD 当请求是来自Web组件转发到另一个组件时,过滤器工作。
\item INCLUDE 当请求来自include操作时,过滤器生效。
\item ERROR 当转发到错误页面时,过滤器起作用。
\end{itemize}
\end{description}
\end{frame}

\begin{frame}[fragile] % [fragile]参数使得能够插入代码
\frametitle{过滤器生命周期} 

\begin{enumerate}\kai
\item {\hei\Blue 创建阶段} 根据<filter-class>标记定义的过滤器类,将类定义加载到服务器内存,并调用此类的默认构造方法,创建过滤器对象。
\item {\hei\Blue 初始化阶段} 创建FilterConfig对象,调用过滤器的init()方法,传入FilterConfig象,完成初始化工作。
\item {\hei\Blue 过滤服务阶段} 每次请求符合过滤器配置的URL时,过滤方法都将执行一次。
\item {\hei\Blue 销毁阶段} 当Web应用卸载或Web容器停止之前,destroy()方法被Web容器调用,完成卸载操作,Web容器销毁过滤器对象。
\end{enumerate}
\end{frame}

\section{过滤器的主要任务}

\begin{frame}[fragile] % [fragile]参数使得能够插入代码
\frametitle{处理HTTP请求} 

\wxd{修改请求头}

调用ServletRequest或HttpServletRequest的各种set请求头方法对请求头进行修改。

\begin{javaCode}
request.setCharacterEncoding("code");  
\end{javaCode}

{\kai 对文字乱码问题,可以在过滤器中进行集中处理,避免在每个Servlet或JSP中进行请求字符编码的转换。}

\wxd{修改请求对象的属性}

调用请求对象的setAttribute方法对请求对象的属性进行增加、修改和删除。

\begin{javaCode}
request.setAttribute("infoType", "image/jpeg"); //设定请求对象的一个属性
request.removeAttribute("userId");
\end{javaCode}

\end{frame}

\begin{frame}[fragile] % [fragile]参数使得能够插入代码
\frametitle{处理HTTP响应} 

过滤器可以在HTTP响应到达客户端浏览器之前,对响应头和响应体进行转换、修改等操作。

过滤器实现对响应处理的代码要在FilterChain的doFilter()方法之后完成,而对请求处理的代码要在doFilter()方法之前进行,流程如下:

\begin{javaCode}
public void doFilter(ServletRequest req, ServletResponse res, FilterChain chain) 
    throws IOException, ServletException {
  ...  // 处理请求的代码放在 doFilter 之后
  chain.doFilter(req, res); // 传递到链中的下个过滤器
  ...  // 处理响应的代码放在过滤器链传递之后
}  
\end{javaCode}
\end{frame}

\begin{frame}[fragile] % [fragile]参数使得能够插入代码
\frametitle{处理HTTP响应} 

\wxd{修改响应头}

\begin{javaCode}
response.setContentType("application/pdf");
response.setContentType("GBK");
\end{javaCode}

\wxd{修改响应体内容}

过滤器使用HTTP响应对象的包装类javax.servlet.http.HttpServletResponseWrapper可以将响应体内容进行重新包装和处理,使浏览器接收到的是经过过滤器处理修改过的响应数据。

见本章后续示例。
\end{frame}

\begin{frame}[fragile] % [fragile]参数使得能够插入代码
\frametitle{阻断HTTP请求} 

常用于用户登录验证等操作。

在某种条件下,实现对请求的阻断,不让请求传递到链中的下一个对象,只要在过滤器的过滤方法中,不执行FilterChain的传递方法doFilter()即可。

\begin{javaCode}
if ( 某条件成立 ) {
  chain.doFilter(request, response); // 继续传递请求
} else {
  response.sendRedirect("url"); // 阻断请求,直接重定向到指定的 URL
}
\end{javaCode}
\end{frame}

\section{本节习题}

\begin{frame}
  \frametitle{本节习题}

  \tta{简答题}
  
  \begin{enumerate}
  \item 什么是过滤器?过滤器通常用于那些场景?
  \item Java EE过滤器的工作原理是怎样的?
  \end{enumerate}

  \tta{小编程}

  \begin{enumerate}
  \item 编写过滤器,结合Session等技术,实现用户登录验证。
  \end{enumerate}
\end{frame}

% TKS %%%%%%%%%%%%%%%%%%%%%%%%%%%%%%%%%%%%%%%%%%%%%%
\begin{frame}
\centering
{\Huge \textcolor{blue}{THE END}} \\
\vspace{5mm}
{\Large wangxiaodong@ouc.edu.cn} \\
\end{frame}
%%%%%%%%%%%%%%%%%%%%%%%%%%%%%%%%%%%%%%%%%%%%%%%%%%
\end{document}
